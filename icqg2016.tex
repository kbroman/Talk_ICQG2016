\documentclass[14pt,t]{beamer}
\usepackage{graphicx}
\setbeameroption{hide notes}
\setbeamertemplate{note page}[plain]
\usepackage{listings}

\input{header.tex}

%%%%%%%%%%%%%%%%%%%%%%%%%%%%%%%%%%%%%%%%%%%%%%%%%%%%%%%%%%%%%%%%%%%%%%
% end of header
%%%%%%%%%%%%%%%%%%%%%%%%%%%%%%%%%%%%%%%%%%%%%%%%%%%%%%%%%%%%%%%%%%%%%%

% title info
\title{Big data in genetics}
\date{}


\begin{document}

% title slide
{
\setbeamertemplate{footline}{} % no page number here
\frame{
  \titlepage

\vfill
\small {\lolit Slides:} \href{http://bit.ly/ICQG2016}{\tt \small
  \color{foreground} bit.ly/ICQG2016}
\hfill
\includegraphics[height=6mm]{Figs/cc-zero.png} \vspace*{-1cm}

  \note{These are slides for a 10-min introduction on Big data in
    genetics, given at the International Conference on Quantitative
    Genetics on 16 June 2016.

    Source: {\tt https://github.com/kbroman/Talk\_ICQG2016} \\
    Slides: {\tt http://bit.ly/ICQG2016\_nonotes} \\
    With notes: {\tt http://bit.ly/ICQG2016}
}
} }


\begin{frame}[c]{}

  \large
  \bbi
\item More samples
\item More phenotypes
\item More genomic-y things
  \ei

\bigskip \bigskip

  \onslide<1|handout 0>{}
  \onslide<2>{\centerline{But {\hilit more} is not necessarily {\hilit better}.}}

  \note{What is big data? It's more data.

    But more is not necessarily better. The effort to measure more
    stuff on more things can lead us towards cheap measures that give
    crap data.

    On the other hand, there are huge opportunities here. I'm
    particularly excited about intermediate, refined phenotypes that
    may have simpler genetic architecture and get us closer to mechanisms.
  }

\end{frame}




\begin{frame}[c]{}

  \large \centerline{Moving data around can be a feat}

  \note{
    Analysis results often considerably bigger than data.

    Saving results to disk can take as long as the calculations
    themselves.

    I recently said to a collaborator, ``I'll just copy this to a USB
    stick.'' But it was going to take an hour. Needed to use an
    ethernet cable instead.
  }

\end{frame}



\begin{frame}[c]{}

  \large \centerline{Better data visualizations}

  \note{
    With big data (and big results), we rely more on data
    visualizations, and we need to do a better job at this.
  }

\end{frame}


\begin{frame}[c]{Pie charts suck}

  \note{
    Pie charts are ineffective because humans are terrible at
    quantifying areas.
  }

\end{frame}


\begin{frame}[c]{Bar charts suck}

  \note{
    Reducing results to two numbers is almost always bad.
  }

\end{frame}

\begin{frame}[c]{Tables suck}

  \note{
    We mostly care about qualitative differences; tables are terrible
    for that.
  }

\end{frame}

\begin{frame}[c]{Tools matter}


  \bbi
\item Need better than {\hilit toy} implementations.
\item No more ``The attached is similar to the code we used.''
\item Work together on common tools.
  \ei

  \note{
    Too often, we focus on the methods and write toy software
    implementation that are sufficient for the methods paper but not
    for anything else.

    Often we don't provide any software for our new methods. And we
    may not take sufficient care in ensuring the computational
    reproducibility of our work.

    And main academic incentives are to make a new tool rather than to
    contribute to others' tools. Working together on common tools
    would be more useful for the community.
    }

\end{frame}



\begin{frame}[c]{Recognize tool makers}


  \bbi
\item Novelty of methods isn't everything
\item Need a home for tool makers in academics
  \ei

  \note{
    Academics (tenure, grants, awards) rewards novel methods far more
    than useful software.

    Folks interested in tool development need to devote considerable
    effort to things they don't care about, or they'll leave for a
    lucrative data science industry job, where their talents are more
    strongly rewarded.

    We need to fix this. It's a cultural problem.
    }

\end{frame}



\begin{frame}[c]{Training}


  \bbi
\item Statistical/computational methods
  \bi
\item \href{https://www.biostat.washington.edu/suminst/sisg}{Summer
  Institute in Statistical Genetics}
  \item \href{https://www.jax.org/education-and-learning/education-calendar/2016/october/short-course-on-systems-genetics}{Short Course on Systems Genetics}
  \ei
\item Data manipulation and management
  \bi
\item \href{http://www.datacarpentry.org/}{\tt datacarpentry.org}
  \ei
\item Software engineering
  \bi
\item \href{http://software-carpentry.org/}{\tt software-carpentry.org}
  \ei
  \ei

  \note{
    Academics (tenure, grants, awards) rewards novel methods far more
    than useful software.

    Folks interested in tool development need to devote considerable
    effort to things they don't care about, or they'll leave for a
    lucrative data science industry job, where their talents are more
    strongly rewarded.

    We need to fix this. It's a cultural problem.
    }

\end{frame}

\begin{frame}[c]{Be open}

  \bbi
\item Open data
\item Open software
\item Open manuscripts
  \ei

  \note{
    Openness of data, software, and manuscripts is better for the
    community.
    }

\end{frame}


\end{document}
