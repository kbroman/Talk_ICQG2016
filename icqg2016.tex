\documentclass[12pt,t]{beamer}
\usepackage{graphicx}
\setbeameroption{hide notes}
\setbeamertemplate{note page}[plain]
\usepackage{listings}

\input{header.tex}

%%%%%%%%%%%%%%%%%%%%%%%%%%%%%%%%%%%%%%%%%%%%%%%%%%%%%%%%%%%%%%%%%%%%%%
% end of header
%%%%%%%%%%%%%%%%%%%%%%%%%%%%%%%%%%%%%%%%%%%%%%%%%%%%%%%%%%%%%%%%%%%%%%

% title info
\title{Big data in genetics}
\date{}


\begin{document}

% title slide
{
\setbeamertemplate{footline}{} % no page number here
\frame{
  \titlepage

\vfill
\small {\lolit Slides:} \href{http://bit.ly/ICQG2016}{\tt \small
  \color{foreground} bit.ly/ICQG2016}
\hfill
\includegraphics[height=6mm]{Figs/cc-zero.png} \vspace*{-1cm}

  \note{These are slides for a 10-min introduction on Big data in
    genetics, given at the International Conference on Quantitative
    Genetics on 16 June 2016.

    Source: {\tt https://github.com/kbroman/Talk\_ICQG2016} \\
    Slides: {\tt http://bit.ly/ICQG2016\_nonotes} \\
    With notes: {\tt http://bit.ly/ICQG2016}
}
} }


\begin{frame}[c]{}

  \large
  \bbi
\item More samples
\item More phenotypes
\item More genomic-y things
  \ei

\bigskip \bigskip

  \onslide<1|handout 0>{}
  \onslide<2>{\centerline{But {\hilit more} is not necessarily {\hilit better}.}}

  \note{What is big data? It's more data.

    But more is not necessarily better. The effort to measure more
    stuff on more things can lead us towards cheap measures that give
    crap data.

    On the other hand, there are huge opportunities here. I'm
    particularly excited about intermediate, refined phenotypes that
    may have simpler genetic architecture and get us closer to mechanisms.
  }

\end{frame}



\end{document}
